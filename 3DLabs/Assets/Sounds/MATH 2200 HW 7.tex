\documentclass[10pt]{article}
\usepackage[margin=2.5cm]{geometry}  % set margins
\usepackage{amsmath}
\usepackage{amssymb} % for special symbols like \mathbb{...}
\thispagestyle{empty} % to prevent page numbering

\begin{document}


\begin{center}
\textbf{MATH 2200 Homework 7}

\textbf{Due}: 11/3 at 11:59pm
\end{center}

Section and problem numbers are from the textbook. A subset of the problems will be graded for accuracy.

\begin{enumerate}

\item Let $f:\mathbb{R}\setminus \{0\}\to \mathbb{R}$ be defined by $f(x) = x + \frac{1}{x}$. Find:
\begin{enumerate}
  \item $f([1,5])$
  \item $f^{-1}((3,4])$
\end{enumerate}

\item Let $f:A\to B$ be a function, let $C,D\subseteq A$ and let $E,F\subseteq B$.
\begin{enumerate}
  \item Prove $f(C\cap D)\subseteq f(C)\cap f(D)$.
  \item Show by example that the inclusion in part (a) does not have to be an equality.
  \item Prove $f^{-1}(E\cup F) = f^{-1}(E)\cup f^{-1}(F)$.
\end{enumerate}

\item Find a one-to-one correspondence between each of the following pairs of sets.
\begin{enumerate}
  \item $\{x,y,\{a,b,c\}\}$ and $\{14,-3,t\}$
  \item $2\mathbb{Z}$ and $17\mathbb{Z}$
  \item $\mathbb{N}\times \mathbb{N}$ and $\{a+bi\in \mathbb{C}\,:\, a,b\in \mathbb{N}\}$
  \item $\mathbb{N}$ and $\{\frac{m}{n}\,:\, m\in \mathbb{N},\, n=1,2\}$
\end{enumerate}	

\item (3.3.7): Suppose $S$ is a (finite) set containing at least two elements, and that for
$A,B\in \mathcal{P}(S)$, we define $A\preceq B$ to mean that $|A|\leq |B|$. Is this relation a
partial order on $\mathcal{P}(S)$? Explain.

\item (3.3.12(c)): Prove that the intervals $(a,b)$ and $(c,d)$ have the same cardinality, where
$a < b$ and $c < d$.

\item (3.3.24):
\begin{enumerate}
\item Let $S$ be an infinite set and let $x$ be an element not in $S$. Prove that $S$ and
$S\cup \{x\}$ are sets of the same cardinality. (You may assume that $S$ contains a
countably infinite subset).
\item Prove that the open and half-open intervals $(0,1)$ and $(0,1]$ have the same cardinality.
\end{enumerate}

\item Let $S$ be the set of all real numbers in the interval $(0,1)$ whose decimal expansions
involve only $0$ and $1$. Prove that $S$ is uncountable. \emph{Note}: This is equivalent to
saying there are an uncountable number of infinite binary strings.

\end{enumerate}
\end{document}
